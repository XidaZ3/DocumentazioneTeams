\section{Introduzione}
\subsection{Scopo del documento}
Questo documento fornisce le informazioni necessarie per l'estensione e la manutenzione del prodotto ShopChain, sono infatti riportate le tecnologie e le scelte progettuali effettuate, in modo che un futuro sviluppatore sappia com'è stato realizzato il prodotto e i prerequisiti necessari per utilizzarlo.
Sono state illustrate anche le procedure per l’installazione per lo sviluppo in locale.

Questo documento fornisce le informazioni necessarie per l'installazione e l'uso del prodotto, e spiega come accedere alle funzionalità da parte dell'utente. Vengono inoltre descritti i requisiti minimi di sistema necessari per il funzionamento corretto di ShopChain.

\subsection{Obiettivi del prodotto}

Al giorno d’oggi, numerosi sono gli e-commerce che non permettono di effettuare transazioni sicure con pagamenti in criptovalute. Infatti, ad esempio, l’acquirente puotrebbe venire truffato dal venditore se dopo il pagamento non gli venisse consegnato il prodotto o viceversa. 

ShopChain è un applicativo in grado di affiancare un e-commerce nelle fasi di pagamento fino alla consegna usando la tecnologia delle blockchain. La blockchain è incaricata di ricevere l’ammontare speso dall’acquirente in criptovaluta, consegnandola al venditore solo quando il pacco gli viene recapitato. 

 Al momento della consegna del pacco l’acquirente dovrà necessariamente inquadrare il QR code applicato sul collo che ne certifica l’avvenuta consegna. Verrà quindi effettuato il passaggio della criptovaluta dallo smart contract della piattaforma al wallet del venditore. 
 
 \pagebreak
 \subsection{Riferimenti}
 
 \subsubsection{Riferimenti informativi}
 \begin{itemize}
     \item  È stato creato il documento \textit{Glossario\_1.0.0.pdf} per chiarire il significato dei termini tecnici che possono creare dubbi e perplessità. 
     \item  La pianificazione è divisa in sprint, seguendo la metodologia agile. Le modalità e il modello di sviluppo sono riportate nel documento \textit{NormeDiProgetto\_1.0.0.pdf}
 \end{itemize}